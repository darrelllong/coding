\documentclass[11pt]{article}
\usepackage{fullpage,fourier,euler,amsmath,graphicx,xspace}
\usepackage{epigraph}
\usepackage{xcolor}
\usepackage{listings}

\title{CSE 13S Coding Standards}
\author{Prof. Darrell Long \& Assistants}
\date{Spring 2021}


\input{../TeX/footer}
\input{../TeX/lststyle}


\begin{document}\maketitle

\epigraphwidth=0.8\textwidth
\epigraph{It is cold in the scriptorium, my thumb aches. I leave
this manuscript, I do not know for whom; I no longer know what it
is about: \emph{stat rosa pristina nomine, nomina nuda tenemus.}}{---Adso of
Melk (Umberto Eco, \emph{Il nome della rosa})}

\section{Introduction}

It is absolutely vital to have coding standards in order to manage complexity
since it's much harder to maintain and understand code that is inconsistent and
lacks documentation. Code that is unintuitive or lacks obvious meaning requires
time to understand. This impairs the process of adding on to the same code
later on, wasting time on merely deciphering what the original code did.
This document defines the coding standards and guidelines that
will be used for CSE\,13S. \textbf{To get full points on your assignments, the
coding standards must be followed completely.} Your assignments are expected to
compile using \texttt{clang} and must be formatted using \texttt{clang-format}.
The \texttt{clang-format} file that you will use can be found under Canvas
files. To use this file to format your source and header files, make sure it
exists in the same directory as your source and headers and run:

\begin{codelisting}{}
clang-format -i -style=file *.c *.h
\end{codelisting}\\


\noindent Custom style configurations can be specified directly using the command line option
\texttt{-style=\emph{<style>}}. However, we use \texttt{-style=file} here to direct \texttt{clang-format}
to look for the style configuration in the \texttt{.clang-format} file present in the same directory.

More in depth information about the \textbf{C} programming language can be found
in the books \emph{The C Programming Language, Second Edition} by Brian
Kernighan and Dennis Ritchie (often called K\&R or Kernighan and Ritchie) and
\emph{C in a Nutshell, Second Edition} by Tony Crawford and Peter Prinz.

\section{General Principles}

\epigraphwidth=0.7\textwidth
\epigraph{\emph{I have deep faith that the principle of the universe will be
beautiful and simple.}}{---Albert Einstein}\noindent

\begin{enumerate}
  \item \emph{Simplicity} and \emph{clarity} should be your goals; do not try to
    be overly clever.

  \item \emph{Global variables} are generally discouraged; you may use them only
    if there is a \emph{very good reason} to do so, such as sharing variables
    between files, storing ``constants'' like command line inputs, or making
    code clearer if it's being obfuscated by an enormous number of parameters
    being passed from function to function just to avoid the use of global
    variables.

  \item \emph{Static variables} outside of functions are slightly less
    proscribed than global variables since their visibility is confined to the
    file where they are defined; that being said, use them only if there is a
    \emph{very good reason} to do so, such as making a value persist across
    function calls.

  \item The \emph{scope} of a variable should be kept as \emph{small} as possible:
    defined and used locally.

  \item Loop control variables (\texttt{for} loops) should be defined in the
    statement so that you are not tempted to rely on their final value.

  \item Variable names, function names, \emph{etc}., must be sensible and
    reflect their purposes. Avoid excessively long variable names. The standard
    variable naming convention for this class will be \emph{snake case} (more on
    this later).

  \item \emph{All} functions require the \texttt{return} keyword, even if the function
    return-type is \texttt{void}.

  \item Avoid excessive and uninformative comments: ``set x to 0'' is not
    informative, unless there is an unclear reason why it needs to be set to 0.
    Comments should explain what the goal is and, if necessary, why.  If your
    code is so complex to the point that an explanation is needed for why it
    works, consider refactoring it into smaller components. At the same time, \emph{do not}
    artificially chop it into smaller pieces.

  \item Indentation should be done using \emph{four} spaces. \textbf{Do not use tabs}.
    Most, if not all, text editors support converting tabs to spaces, as well as
    setting the default tab-width.

  \item Line width should be limited to 100 characters. This avoids having
    excessively long lines of code and forces you to be more clear and concise.

  \item Line breaks, or blank lines, in code should be used to separate lines of
    code to elucidate which lines of code are related or are used in conjunction
    for some purpose.

  \item Above all else, be \emph{consistent} in your coding style.
\end{enumerate}


\section{Compilation, External code, and Makefiles}

All assignments are expected to compile using \texttt{clang} on your Ubuntu
20.04 virtual machine using the following command.

\begin{codelisting}{}
clang -Wall -Wextra -Wpedantic -Wshadow -std=c17 <filename>
\end{codelisting}

\noindent Code must compile without warnings or errors, return no
\texttt{scan-build} errors, and pass a \texttt{valgrind} leak check to recieve
full credit.
Code obtained from external sources such as \texttt{stackoverflow.com} must be
cited using a comment block placed at the top of the file and before the code. The comment must specify
the copied lines, the original author, and a link to the source. You will not
recieve credit for code that is not your own. Functions and interfaces provided
by the \textbf{C} standard library may be used unless otherwise specified by the
instructors and/or assignment.

A \texttt{Makefile} must be included with each assignment submission and should
contain the following targets.  Additional targets can be included but should be
detailed in an assignment's \texttt{README}.
\begin{enumerate}
  \item \texttt{all}: The default target that builds the program.
  \item \texttt{clean}: Removes build artifacts, such as \texttt{*.o} files, and the executable.
\end{enumerate}

\section{scan-build}

\texttt{scan-build} is a command line utility which allows you to run the \texttt{clang} static analyzer on your code.
The static analyzer essentially finds bugs in your code which is really helpful when it comes to large projects
which can have a lot of files and thousands of lines of code. However, only files which have been compiled are
analyzed. To use the analyzer from the command line use the \texttt{scan-build} command before your build command
as shown.

\begin{codelisting}{}
scan-build make
\end{codelisting}


\section{Formatting and Naming}

\epigraphwidth=0.85\textwidth
\epigraph{\emph{
In hanc utilitatem clementes angeli saepe figuras, characteres, formas et voces
invenerunt proposuerunt que nobis mortalibus et ignotas et stupendas nullius rei
iuxta consuetum linguae usum significativas, sed per rationis nostrae summam
admirationem in assiduam intelligibilium pervestigationem, deinde in illorum
ipsorum venerationem et amorem inductivas.}}{---Johannes Reuchlin, De arte
cabalistica, Hagenhau, 1517, III}


\subsection{Braces, Parentheses, and Spaces}

Keywords and whatever follows should be separated by a space except for
semicolons. Function calls have \emph{no space} between
the arguments and the function name.
Curly braces surrounding a function or control structure's code block will have
the opening curly brace on the same line as the function or control structure's
definition. Objects enclosed within parentheses should not include padding spaces.

\begin{codelisting}{}
while (i < 5) {           //control structure
    printf("i is %d\n", i); //function call
    i++;
}
\end{codelisting}

Control structure keywords, such as \texttt{while}, \texttt{if},
\texttt{switch}, \texttt{for}, \emph{etc}., which appear at the start of code
blocks, should always have opening and closing curly braces. Even if the
contained statement is only one line. This provides a more consistent look,
avoids possible bugs introduced while editing code, and makes it easier to add
code between the braces later on.

\begin{codelisting}{}
if (i < 5) {
    puts("This style is good!");
}

if (i > 5)
    puts("This style is bad!");
\end{codelisting}


\subsection{Variable and Function Naming}

Variable and function names must be \emph{snake cased}. This means that all
characters are lower case and all words are separated by underscores.  Variable
and function names should be concise and clearly describe their function and/or
purpose. Exceptions are for iteration control variables, in which they can be
single characters (such as \texttt{i}, \texttt{j}, and \texttt{k}), and
temporary variables (such as \texttt{temp}).

\begin{codelisting}{}
// Example of snake case variable naming with a loop.
uint32_t odd_count = 0;
for (uint32_t i = 0; i < 4096; i += 1) {
    if (i % 2 != 0) {
        odd_count += 1;
    }
}
\end{codelisting}

In the case of pointers, the '*' is directly followed by the variable name.
malloc returns type (void *), so we cast it to type (char *).

\begin{codelisting}{}
char *first_name = (char *) malloc(10 * sizeof(char));
char *last_name = (char *) malloc(10 * sizeof(char));
\end{codelisting}


\subsection{Structure Naming}

% TODO Typedef'd structs may be confusing for new programmers
Structure names will be \emph{Pascal cased}. Pascal case is a subset of \emph{camel
case} in which the first letter is also capitalized. Each \texttt{struct} should
be \texttt{typedef}'d to create opaque, exportable types.

\begin{codelisting}{}
// Typedef struct TreeNode to be just TreeNode.
typedef struct TreeNode TreeNode;

//
// Definition of struct TreeNode, a binary tree node.
//
// left:   The node's left child.
// right:  The node's right child.
// data:   The node's data.
//
typedef struct {
    TreeNode *left;
    TreeNode *right;
    uint32_t data;
} TreeNode;
\end{codelisting}


\section{Variables and Types}

\epigraphwidth=0.35\textwidth
\epigraph{\emph{When my code has tons of trouble\\ Wesley Mackey comes to me\\
Speaking words of wisdom: \\ ``Code in C''}}{}


\subsection{Booleans}

\noindent Booleans should be of type \texttt{bool} and not \texttt{int}.  All
boolean values in \textbf{C} are technically integers where $0$ is
\texttt{false} and anything else is considered \texttt{true}. Although
\texttt{bool} and \texttt{int} are functionally interchangeable when evaluating
conditionals, using these different types communicates a specific variable's
purpose and therefore clarifies otherwise vague code. The \texttt{bool} type is
provided by the header file \texttt{stdbool.h}.

For example, assume a function named \texttt{pop()} that pops an item off a
stack. Depending on the implementation, \texttt{pop()} could either return the
removed value or an exit status. For this example, assume the purpose of the
function is to remove the item and return true if the item was removed
successfully and false if it wasn't.

\begin{codelisting}{}
// Not using a boolean makes the purpose of the function
// unclear.
int pop(Stack *s, int *top);

// Using a boolean makes the purpose of the function clearer.
bool pop(Stack *s, int *top);
\end{codelisting}


\subsection{Integers}

\noindent Include \texttt{stdint.h} or \texttt{inttypes.h} (the latter is a
superset of the former) and use the fixed-width integer types for integer
variables.  These are defined as \texttt{intN\_t} and \texttt{uintN\_t}, where
\texttt{N} is either 8, 16, 32, or 64, and refers to the width of the integer in
bits (\emph{e.g.} \texttt{uint64\_t} refers to a 64-bit unsigned integer). You
should only use signed integers only if you need negative values.

To maximize portability of passing integers to functions that operate on
\texttt{printf}-like format strings, you should include \texttt{inttypes.h} to
use defined format specifier macros. These macros are defined like
\texttt{PRI\textbf{xN}}, where, \texttt{\textbf{x}} is a \texttt{printf} format
specifier like \texttt{d} or \texttt{c}, and \texttt{\textbf{N}} is the width in
bits of the variable.

\begin{codelisting}{}
// Example of printing an uint64_t with unsigned integer
// formatting.
uint64_t n = 42;
printf("n equals %" PRIu64 "\n", n);

// Printing an int32_t with decimal formatting.
int32_t m = 360;
printf("m equals %" PRId32 "\n", n);

// Printing an integer in hexadecimal.
printf("%" PRIx64 "\n", n);

// Printing an integer in octal.
printf("%" PRIo64 "\n", n);
\end{codelisting}


\subsection{Floats and Doubles}

\noindent Both floats and doubles, denoted by \texttt{float} and \texttt{double}
respectively, are used to represent floating point numbers. You can think of
floating point numbers as real numbers with limited precision. A \texttt{double}
has twice the precision of a \texttt{float}. However, since we are basically
squeezing infinitely many real numbers into a finite number of bits, floating
point numbers can only give an approximate representation, and thus are rounded.
Due to this rounding, there can be errors in certain calculations and those errors
can add up and snowball into very large errors. So how do you pick between using a \texttt{double} and a
\texttt{float}? \emph{Use as much precision as is needed and nothing more.} A
\texttt{float} can represent up to 7 decimal digits, and a \texttt{double} up to
15 decimal digits. If you do not need more than 7 digits, use a \texttt{float}.
When declaring a floating point number, add a `\texttt{.0}' after the significant to
indicate that the value is not an integer.

\begin{codelisting}{}
// Example of declaring pi as a double and float.
#define PI 3.141592653589793238462643383279502884197

double life = 42.0 // Include the '.0' to show it's not an integer.
float pi_float = PI;
double pi_double = PI;

printf("Pi as float: %0.20f\n", pi_float);
// This gives up to 7 decimal points of precision.
// Output: "Pi as float: 3.14159274101257324218"

printf("Pi as double: %0.20lf\n", pi_double);
// This gives up to 15 decimal points of precision.
// Output: "Pi as double: 3.14159265358979311599"
\end{codelisting}


\subsection{Constants}

The keyword \texttt{const} serves as a type \emph{qualifier}. In the type
\texttt{int const}, \texttt{int} is the type \emph{specifier}, and
\texttt{const} is the type \emph{qualifier}. You should always write the qualifier after the type in
which it qualifies.  The purpose of \texttt{const} is to restrict the value of a
variable to solely what it is initialized to. You can't assign a value to a
\texttt{const} value after it has already been declared, even if the value to
assign it is the same value it was declared with. You also can't assign a value
to a \texttt{const} value even if it wasn't initialized to anything when it was
first declared.

\begin{codelisting}{}
int const foo = 5;  // This is fine.
foo = 6;            // This is not fine.
foo = 5;            // This is still not fine.
\end{codelisting}

\noindent \texttt{const} pointers are a bit more difficult. Where the \texttt{*} appears
in the declaration determines what is considered constant, the pointer or the
data it is pointing two. In the case of \texttt{int const *foo} we have a
pointer to a \texttt{int const} type. Here the pointer can be modified but the
data it points to cannot. In the case of \texttt{int *const foo} we have a
constant pointer to an \texttt{int}. The data can be modified but the pointer
cannot. For this reason one must be careful when declaring \texttt{const}
pointers.

Constants are described in more detail in \S 2.4 of Kernighan and Ritchie and
Chapter 11 of \emph{C in a Nutshell}.

\begin{codelisting}{}
int const *foo; // pointer to an int const, the pointer
                // can be modified.
int *const foo; // const pointer to an int, can
                // modify the data.
\end{codelisting}


\subsection{\texttt{static}}

There are two uses for the \texttt{static} keyword in \textbf{C}. A static
variable is declared \emph{inside} a function if the value of the variable needs
to persist across function calls and should only exist within the scope of the
function it's declared in. A static variable is declared \emph{outside}, if the
value of the variable needs to persist across function calls and should only
exist within the scope of the file where it's declared.  Functions and global
variables that are not accessed outside of the file where they are defined
should be declared \texttt{static}.

The example shown below demonstrates how \texttt{static} maintains the value of
statically declared variables across function calls.

\begin{codelisting}{\texttt{static.c}}
static uint32_t x = 0; // Can only be accessed within static.c

void increment() {
    // This value will persist across function calls.
    static uint32_t y = 0;

    x += 1;
    y += 2;

    printf("%d ", x);
    printf("%d\n", y);
}

void main() {
    increment(); // 1 2 should be printed.
    increment(); // 2 4 should be printed.
}
\end{codelisting}

More information about \texttt{static} variables and functions can be found in
\S 4.6 of Kernighan and Ritchie and Chapters 7 and 11 of \emph{C in a Nutshell}.


\subsection{\texttt{extern}}

The \texttt{extern} keyword extends the visibility of variables and functions so
that they can be called from any one of the program's files, \emph{provided that
the declaration is known.} By default, functions in \textbf{C} implicitly
prepend an \texttt{extern} keyword and hence extern is not used in function declarations and definitions.
Explicit use of the \texttt{extern} keyword generally revolves around variables that to for variable declarations.
These variables are essentially, global variables. Variables can be declared any number of times but they can
be defined only once.  By defining a variable we mean that memory is allocated to the variable. In the following
example, \texttt{util.c} defines a variable declared as \texttt{extern} in \texttt{util.h}. A file that includes \texttt{util.h}
will be able to see and modify the \texttt{extern} variable.

\begin{codelisting}{\texttt{util.h}}
#ifndef __UTIL_H__
#define __UTIL_H__

#include <stdbool.h>

extern int counter; // External counter declaration.

#endif __UTIL_H__
\end{codelisting}

\begin{codelisting}{\texttt{util.c}}
#include <stdio.h>
#include "util.h"

int counter = 42; // Counter definition.

int decrement(void) {
    return counter--;
}
\end{codelisting}

\texttt{extern} declarations and scoping rules are described more in \S 1.10 and
\S 4.4 of Kernighan and Ritchie and Chapters 7 and 11 of \emph{C in a Nutshell}.


\subsection{Type Casting}

When an operation has operands of different types, \textbf{C} will try to
convert the operands into a common type. In general, these automatic type
conversions are those that convert a ``narrower'' type into a ``wider'' type. An
example of this would be converting a \texttt{uint8\_t} into a
\texttt{uint32\_t}, the former of which is smaller than the latter. Another
example is in the case of integer division. In \textbf{C} integer division
truncates, or returns an integer and discards the remainder.  Casting to a
\texttt{float} or a \texttt{double} as shown in the example below will prevent
truncation and return an accurate value. One must also take care to not cast
from a double to a float as this will result in lost accuracy.

In \textbf{C}, explicit type conversions can be forced using the unary
\emph{cast} operator. Type casting is done through the construction of
(\emph{type-name}) \emph{expression}. The following are examples of explicit
type casting:

\begin{codelisting}{}
// Converts a char into a uint8_t.
char a = 'a';
uint8_t c = (uint8_t) a;

// Avoiding truncation with integer division
double f;
f = 1/2;
// This results in f=0.0. Since the operation was performed
// as integer division the quotient was truncated and only
// stored as a double.

f = (double) 1/2;
// This gives us f=0.5 as one of the two operands were cast
// to a double.
\end{codelisting}

Type casting should be used whenever you allocate memory using
\texttt{malloc}-like functions, like in the second example above. This is
because \texttt{malloc}-like functions return \texttt{(void *)}'s. Other than
for casting \texttt{(void *)}'s, type-casting should be done rarely and you
should be careful with operator precedence if you must type-cast.

\begin{codelisting}{}
// Allocating memory for an int array.
int *arr = (int *) malloc(10 * sizeof(int));
\end{codelisting}

Type conversions and casting is described in \S 2.7 and \S A6 of Kernighan and
Ritchie and Chapter 4 of \emph{C in a Nutshell}.


\subsection{Enumeration}

Enumeration of constants, or assigning names to constants, is done using
\texttt{enum}. Enumeration is typically used for a set of related symbols to
give a unique value. A good example would be enumerating status codes. You
should never enumerate magic numbers; always define magic numbers using
\text{\#define}. Enumerations should be \texttt{typedef}'d. If the enumeration
tokens aren't manually given a value, the first token will by default be
assigned the value of 0, the next token assigned the value of 1, and so on and
so forth. You can also specify what value to start enumerating from, and if you
wish, manually assign values for each \texttt{enum} token.

\begin{codelisting}
// Example of enumerating a week.
// Tue == 3, Wed == 4, etc.
typedef enum { Mon = 2, Tue, Wed, Thu, Fri, Sat, Sun } Week;

// How to use the enumeration.
Week day = Mon;
printf("%d\n", day); // This prints out 2.
\end{codelisting}

Enumerations are described in \S 2.3 of Kernighan and Ritchie and Chapter 2 of
\emph{C in a Nutshell}.


\section{Control Structures}

\epigraphwidth=0.35\textwidth
\epigraph{\emph{Code in C, Code in C, Code in C, ...\\ There will be an answer,
Code in C.}}{}


\subsection{If Statements}

If-else statements directly follow one another. This means that that an
\texttt{else} or \texttt{else if} statement following an \texttt{if} statement
should be on the same line as the \texttt{if} statement's closing brace.

\begin{codelisting}{}
if (first condition) {
    // Statements if first condition is true.
} else if (second condition) {
    // Statements if second condition is true.
} else {
    // Statements if neither are true.
}
\end{codelisting}

If-else and Else-if are described further in \S 3.2 and \S 3.3 of Kernighan and
Ritchie and Chapter 6 of \emph{C in a Nutshell}.


\subsection{Loops}

Infinite loops are to be written using \texttt{while} loops. Infinite loops
should be exited using \texttt{break}. In reality, \texttt{goto} can also be
used to since \texttt{goto}, like \texttt{break}, alters the normal execution of
code, but using it without very good reason to do so is considered bad practice
as explained in Dijkstra's letter, ``Go To Statement Considered
Harmful''\footnote{Edsger W. Dijkstra. 1968. Letters to the editor: go to
  statement considered harmful. Communications of the ACM 11, 3 (March 1968),
147--148.}. Kernel programming sometimes uses \texttt{goto} for error handling,
but you are not programming in the Kernel.

\begin{codelisting}{}
// Make sure to use the stdbool.h constant "true".
while (true) {
    printf("Deja vu");
}

// Example of a for loop.
for (int i = 0; i < 5; i++) {
    foo++;
}

// Example of a do-while loop.
do {
    foo++;
} while (i < 5);
\end{codelisting}

 More information on loops can be found in \S 3.5 of Kernighan and Ritchie and
Chapter 6 of \emph{C in a Nutshell}.


\subsection{Switch Statements}

Switch statements are an exception to the previously describe brace conventions.
Although the switch statement, denoted by \texttt{switch}, does require braces
like a loop or \texttt{if} statement. The encapsulated \texttt{case} statements
\emph{do not} require them. Adding extra braces for clarity does not hurt and
feel free to do so as long as you are consistent. The \texttt{break} statement
must follow \emph{all} statements for a specific case, unless the goal is to
fall through more than one case.  Fall-through is the act of not including the
\texttt{break} statement so that more than one case can share code. You should
generally avoid having a fall-through in your switch statements. All
\texttt{switch} statements should also have a \texttt{default} case, which is
invoked if no case conditions are met, even if it is empty. \emph{We will mark you down if you don't!}

\begin{codelisting}{}
switch (i % 2) {
case 0:
    printf("Variable i (\%d) is even!\n", i);
    break;
case 1:
    printf("Variable i (\%d) is odd!\n", i);
    break;
default:
    puts("How is it neither even nor odd?!");
    break;
}
\end{codelisting}

Switch statements and their specifics are more thoroughly described in \S 3.4 of
Kernighan and Ritchie and Chapter 6 of \emph{C in a Nutshell}.


\section{Includes and Defines}

\epigraphwidth=0.6\textwidth
\epigraph{\emph{Eleanor Rigby \\
Sits at the keyboard and waits for a line on the screen \\
Lives in a dream \\
Waits for a signal \\
Finding some code that will make the machine do some more.  \\
What is it for? \\
All the lonely users, where do they all come from? \\
All the lonely users, why does it take so long?}}{}

\noindent The first step of compiling a \textbf{C} program is running the
preprocessor. The major tasks for the preprocessor are to include header files
specified by \texttt{\#include} and to perform token replacement with
\texttt{\#define}. The features of the preprocessor are more thoroughly
described in \S 4.11 of Kernighan and Ritchie and Chapter 15 of \emph{C in a
Nutshell}.\\

The primary way to connect multiple files together into one program is through
the use of \emph{header files} which are specified with the \texttt{.h} file
extension. The contents of these files are included in your program through the
\texttt{\#include} command. Header files should include prototypes and
declarations for functions, structs, \emph{etc.} that will be visible outside of
their encapsulating \texttt{.c} file. The use of header files is more thoroughly
described in \S \ref{headers} of this document.

The \texttt{\#define} macro tells the preprocessor to replace a string of text
with another before the code is actually compiled. This can be used to give
names to ``magic'' numbers, or numbers who's purpose is not obvious to the
reader.  It is also useful to give names to constants that appear multiple times
in a program such as block sizes or maximum values for a variable. Lastly, some
basic functions or repeated commands can be represented through macros but this
should be used with caution. The \texttt{\#define} macro also enables the
programmer to modify all instances of a specific string or number while only
editing a single line of code.

Includes and defines are to be formatted the same way: no space after the '\#'
character. Thus, includes and defines are formatted as shown in the following
example. The names of \texttt{\#define} macros must be in snake case and all
caps.

\begin{codelisting}{}
// Example of including standard I/O and boolean libraries.
#include <stdio.h>
#include <stdbool.h>
// Example of including one of your own header files.
#include "foobar.h"

// Example of defining a block size macro.
// Every instance of BLOCK_SIZE is replaced with 4096.
#define BLOCK_SIZE 4096

// Example of preventing duplicate macro definitions using #ifndef.
// Macros that go between the #ifndef and #endif are guarded.
#ifndef MAGIC
#define MAGIC 0xdeadd00d
// ... other defines ...
#endif
\end{codelisting}

In general, you must be careful with macros. They use \emph{text substitution}
and do not work through function calls. Some avoid using them as much as
possible. It is our opinion that they should be used with care. A better
alternative to macros that are more complex than a simple value is an
\emph{inline function} as shown below.

\begin{codelisting}{}
// Dangerous under the right circumstances.
#define MAX(X, Y) (((X) > (Y)) ? (X) : (Y))

// A call like this.
MAX(a++, b++)
// Will be replaced with this, which increments the returned
// value twice.
(((a++) > (b++) ? (a++) : (b++))

// A better macro using compiler builtins,
// a bit complex and difficult to read.
// Portability is also of concern.
#define BETTER_MAX(X,Y) \
    ({ __typeof__(X) _X = (X); \
     __typeof__(Y) _Y = (Y);  \
     _X > _Y ? _X : _Y; })

// The best option is an inline function.
inline int max(int a, int b) {
    if (a > b) {
        return a;
    }

    return b;
}

// Another inline funciton.
inline int other_max(int a, int b) {
    return a > b ? a : b;
}
\end{codelisting}

The first macro can produce what is called a side effect, an unintended changed
to a variable. In this case it can lead to the returned value being modified
twice.  The second macro is a bit better but uses some arcane compiler
incantations to create temporary copies of the two variables and therefore can
be difficult to understand. The best solution that you should use is an
\emph{inline function}, which performs the same task with better protections and
similar behavior when compared to a macro. The last example uses the ternary
operator or $?$, which helps write clean and simple code. Two ternary operator
examples are shown below.

\begin{codelisting}{}
// condition ? true : false;
x = a > b ? a-- : b++;
return_code = a == b ? foo() : bar();
\end{codelisting}\\

When in doubt, seek to be easier to understand than clever.


\section{Comments}

\epigraphwidth=0.8\textwidth
\epigraph{\emph{Life should not be a journey to the grave with the intention of arriving safely in a pretty and well preserved body, but rather to skid in broadside in a cloud of smoke, thoroughly used up, totally worn out, and loudly proclaiming ``Wow! What a Ride!''}}
{---Hunter S. Thompson, \emph{The Proud Highway: Saga of a Desperate Southern Gentleman, 1955--1967}}

\noindent You will be using ``\texttt{//}'' style comments, with a single space
separating the ``\texttt{//}'' comment prefix and the comment itself. These are
line comments and only comment out characters that follow it on the same line.
We will \emph{not} be using the other bracket style comment ``\texttt{/* \ldots
*/}'' as it is error-prone.  Again, avoid unnecessary and excessive comments.
Commenting is a balancing act.  You do not want to comment too much or too
little. The goal of is to explain what is being done or what is not immediately
evident.

All functions and structs must also provide a comment block right above their
declarations, briefly explaining their usage, return value, and details of their
inputs and outputs. Basically the following questions must be answered,
"What does the function do?",
"What are its inputs and outputs?" and "Are there any cases I need to watch out for?".
An example of a basic comment block for a function using
``//'' line comments:

\begin{codelisting}{}
//
// Increments an integer that is passed as a parameter. Returns the new value of
// of n on success and -1 on failure.
//
// n:  The unsigned integer to increment.
//
int increment_int(uint32_t *n) {
    if (n != NULL) {
        *n += 1;
        return *n; //return the new value of n
    } else {
        return -1; //return an error
  }
}
\end{codelisting}


\section{Abstract Data Types (ADTs)}

\epigraph{\emph{The public knows that human beings are fallible. Only
people blinded by ideology fall into the trap of believing in their
own infallibility.}}{---Freeman Dyson}

\noindent You will be learning many data structures and making abstract data
types using them this quarter. An ADT is a type, or class, for objects whose
behavior is defined by a set of functions. These functions are split into three
categories: constructor/destructor, accessor, and manipulator. An ADT
constructor handles allocating memory for a new instance of the ADT, its
destructor frees any allocated memory associated with the ADT. Accessor
functions are mainly used in ADT implementations where the design of the ADT is
not made explicit and used to get ADT fields. In this class, struct definitions
of ADTs will go in header files, so this is not an issue. Manipulator functions
serve to alter values stored in an ADT. ADTs should always have a source file
(\textbf{C}-file) and a corresponding header file.


\subsection{ADT Functions}

All functions must have a comment block preceding it that gives a brief overview
on the purpose and usage of not just the function, but also the input
parameters, followed by the output parameters. Return types of functions must be
explicitly stated, even if it is a \texttt{void}-type function. ADT functions
must be written \emph{noun} then \emph{verb} style, where \emph{noun} refers to
the ADT and \emph{verb} the action to perform on the ADT object.


\subsection{ADT Header Files} \label{headers}

Header files serve as ADT interfaces and contain function prototypes and
definitions of functions implemented in their respective corresponding source
files.

Header files must have a header guard before all function prototypes to prevent
duplicate definitions. Header guards are formatted as two underscores followed
by the name of the header it guards, another underscore, an "H", then followed
by another two underscores. Header guards are written using \texttt{\#ifndef},
which reads as "if not defined". Always remember to close a \texttt{\#ifndef}
with a corresponding \texttt{\#endif}. Header files must also have a comment at
the top of the file explaining its purpose with regard to the ADT function
prototypes.

A header file will include the declarations for structs, global variables,
function prototypes, \emph{etc.} that will be visible to other \texttt{.c}
files. Variable definitions, function source code, and the like do not belong in
a header file with the exception of inline functions.


\subsection{ADT Source Files}

These source files contain the implementations of all functions prototyped in
their corresponding header files as well as definitions for their variables.
Declarations of structs and variables and function prototypes that are not to be
visible to other \texttt{.c} files also appear in a source file.  Excluding the
header guard, all code in source files are commented and formatted the same way
as in header files.


\section{Example BitVector ADT}

\epigraphwidth=0.55\textwidth
\epigraph{\emph{
Vector: You wanted to see me dad. \\
Mr. Perkins: Oh yes, come in Victor. \\
Vector: Actually, Victor was my nerd name.  Now its Vector! \\
Mr. Perkins: Sit down!
}}{\emph{Despicable Me}}

\noindent In the given example, the \emph{bv\_create} function is an example of a constructor function and
 \emph{bv\_delete} function is an example of a destructor function.

\begin{codelisting}{\texttt{bv.h}}
// bv.h - Contains the definitions used by the BitVector ADT.

#ifndef __BV_H__ // Check if token has been defined already.
#define __BV_H__

#include <inttypes.h>

//
// A BitVector is a dynamically allocated vector of bits.
//
// vector: An array of bytes to store bits in.
// length: The length in bits of the BitVector.
//
typedef struct BitVector BitVector;

//
// Creates and returns a BitVector.
//
// length: The number of bits the BitVector represents.
//
BitVector *bv_create(uint32_t length);

//
// Frees a BitVector.
//
// bv: The BitVector to free.
//
void bv_delete(BitVector **bv);
\end{codelisting}

\begin{codelisting}{\texttt{bv.c}}
// bv.c - Contains the implementation of the BitVector ADT.

#include <inttypes.h>
#include "bv.h"

struct BitVector {
    uint8_t *vector;
    uint32_t length;
};

//
// Creates and returns a BitVector.
//
// length: The number of bits the BitVector represents.
//
BitVector *bv_create(uint32_t length) {
    BitVector *bv = (BitVector *) malloc(sizeof(BitVector));
    bv->length = length;
    bv->vector = calloc(length / 8 + 1, sizeof(BitVector));
    return bv;
}

//
// Frees a BitVector.
//
// bv: The BitVector to free.
//
void bv_delete(BitVector **bv) {
    free((*bv)->vector);
    (*bv)->vector = NULL; // Always null a freed pointer.
    free(*bv);
    *bv = NULL;
    return; // Return is always required, even for void-type functions.
}
\end{codelisting}

\end{document}
